\documentclass[11pt, a4paper, twocolumn]{article}


%Pacotes:
\usepackage[brazilian]{babel}
\usepackage[utf8]{inputenc}
\usepackage{graphicx} % Inserir Imagem
\usepackage{multicol}
\usepackage{siunitx}
\usepackage{authblk}
\usepackage[top=2cm, bottom=2cm, left=3cm, right=2cm]{geometry}
\usepackage{listings}

\begin{document}

\title{Centro Universitário SENAI CIMATEC\\
    \vspace{1cm}
    \textbf{Relatório de Simulação \\ Webots}
}

\author[1]{Marcelo Borges}
\author[2]{Frederico Barreto}
\author[3]{Júlia Nascimento}
\affil[1]{Engenharia Elétrica - marcelo.borges@aln.senaicimatec.edu.br}
\affil[2]{Engenharia Elétrica - frederico.castellucci@aln.senaicimatec.edu.br}
\affil[3]{Engenharia Elétrica - julia.ribeiro@aln.senaicimatec.edu.br}

\date{Salvador, Bahia\\\today}
\maketitle    
\begin{abstract}
    \textbf{A simulação foi realizada através da utilização do softwere Webots.}
\end{abstract}

\section{Introdução}
    Quando se pensa em sistemas embarcados é preciso entender que o mesmo utiliza de sistemas microprocessados no qual o computador é completamente 
    encapsulado ou dedicado ao dispositivo que ele controla. Os sistemas embarcados são utilizados para executar tarefas específicas em um determinado
    aparelho. Eles estão presentes em diversos equipamentos do dia a dia, como por exemplo em semáforos, aparelhos de ar condicionado (controle da temperatura),
    impressoras, tablets, smartphones e MP3 players.
    \\
    Para essa atividade de simulação, o softwere utilizado foi o Webots, que é um simulador de robô 3D gratuito e opensource usado na indústria, educação e 
    pesquisa. Esse programa inclui uma grande coleção de modelos livremente modificáveis de robôs e objetos. Além disso, também é possível construir novos modelos do zero
    ou importá-los de um software CAD 3D. Ao projetar um modelo de robô, o usuário especifica  as propriedades gráficas e físicas dos objetos. As propriedades gráficas incluem 
    a forma, dimensões, posição e orientação, cores e textura do objeto. As propriedades físicas incluem a massa, o fator de fricção, bem como as constantes de mola e amortecimento. 
    \\
    Por fim, o webots inclui um conjunto de sensores e atuadores frequentemente utilizados em experimentos robóticos como, por exemplo, lidars, radares, sensores de proximidade, 
    sensores de luz, sensores de toque, GPS, acelerômetros, câmeras, emissores e receptores, servo motores (rotacionais e lineares), sensor de posição e força, LEDs, pinças, giroscópio, 
    bússola, etc.

\section{Desenvolvimento}
    A simulação foi baseada em buscar solucuonar o desafio proposta, que foi: manter um robô andando na linha de uma pista e desviar de alguns objetos colocados ao longo dela para 
    que o robô utilizasse dos seus sensores para realizar o desvio correto e após issi voltar para a linha. Com isso, alguns materiais disponibilizados pelo softwere Webots foram 
    necessário para a realização da atividade, foram eles:
    \begin{itemize}
        \item Robô: e-puck;
        \item Pista;
        \item Objetos;
        \item Arena: madeira 1x1m
    \end{itemize}
    Utiliazndo tais componentes, foi preciso criar no simulador um mundo, com dimensões 1x1m, que serve como base geral para que fossem adicionados os outros elementos que fariam parte 
    e viabilizaria a simulação de fato. O primeiro elemento a ser inserido na arena é uma pista fechada contendo aproximadamente 8 partes curvadas, vide figura 01, para que o robô consiga 
    explorar seus sensores para permanecer na linha corretamente.
    \begin{figure}[h]
        \caption{Pista}
        \centering
        \includegraphics[width=7cm]{imagens/pista.png}
    \end{figure}
    \\
    Os objetos foram os próximos elementos a serem inseridos na pista, uma vez que esses seriam os obstáculos para que o robô conseguisse desviar. Os objetos utilizados foram uma esfera (bola), 
    uma caixa de madeira e uma lata de refrigerante. Já o último elemento a ser inserido no mundo para completar a simulação foi o robô e-puck, vide figura 02, que iria fazer o papel principal na 
    atividade. Ou seja, a partir dos elementos secundários o robô conseguiria funcionar de maneira correta andando na pista normalmente e desviando dos objetos inseridos. O e-puck é autônomo e 
    possui 11 sensores IR, 2 motores, câmera colorida 640x480, 8 LEDs em anel, 1 LED de corpo, 1 LED frontal, 3 microfones e 1 autofalante.
    \begin{figure}[h]
        \caption{E-puck}
        \centering
        \includegraphics[width=7cm]{imagens/epuck.png}
    \end{figure}
    \\
    Após a contrução do ambiente da simulação com todos os elementos necessários, foi o momento da codificação e correções do \textit{controler} que iria dar a autonomia e comandos para o E-puck funcionar 
    conforme o desafio. A simulação foi dividida em 3 etapas principais: Simulação do ambiente com o robô e os objetos; Missão 01, percorrer toda a pista; e Missão 02, percorrer toda a pista e desviar dos 
    obstáculos.
    \\
    Seguindo as etapas citadas, a primeira função do \textit{controler} que foi modificada inicialmente foi a \textit{LFM - Line Following Module}, mais especificamente a velocidade ("DeltaS") do robô na 
    pista.

    \section{Conclusão}

\end{document}